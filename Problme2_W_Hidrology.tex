% Options for packages loaded elsewhere
\PassOptionsToPackage{unicode}{hyperref}
\PassOptionsToPackage{hyphens}{url}
%
\documentclass[
]{article}
\usepackage{amsmath,amssymb}
\usepackage{lmodern}
\usepackage{iftex}
\ifPDFTeX
  \usepackage[T1]{fontenc}
  \usepackage[utf8]{inputenc}
  \usepackage{textcomp} % provide euro and other symbols
\else % if luatex or xetex
  \usepackage{unicode-math}
  \defaultfontfeatures{Scale=MatchLowercase}
  \defaultfontfeatures[\rmfamily]{Ligatures=TeX,Scale=1}
\fi
% Use upquote if available, for straight quotes in verbatim environments
\IfFileExists{upquote.sty}{\usepackage{upquote}}{}
\IfFileExists{microtype.sty}{% use microtype if available
  \usepackage[]{microtype}
  \UseMicrotypeSet[protrusion]{basicmath} % disable protrusion for tt fonts
}{}
\makeatletter
\@ifundefined{KOMAClassName}{% if non-KOMA class
  \IfFileExists{parskip.sty}{%
    \usepackage{parskip}
  }{% else
    \setlength{\parindent}{0pt}
    \setlength{\parskip}{6pt plus 2pt minus 1pt}}
}{% if KOMA class
  \KOMAoptions{parskip=half}}
\makeatother
\usepackage{xcolor}
\usepackage[margin=1in]{geometry}
\usepackage{color}
\usepackage{fancyvrb}
\newcommand{\VerbBar}{|}
\newcommand{\VERB}{\Verb[commandchars=\\\{\}]}
\DefineVerbatimEnvironment{Highlighting}{Verbatim}{commandchars=\\\{\}}
% Add ',fontsize=\small' for more characters per line
\usepackage{framed}
\definecolor{shadecolor}{RGB}{248,248,248}
\newenvironment{Shaded}{\begin{snugshade}}{\end{snugshade}}
\newcommand{\AlertTok}[1]{\textcolor[rgb]{0.94,0.16,0.16}{#1}}
\newcommand{\AnnotationTok}[1]{\textcolor[rgb]{0.56,0.35,0.01}{\textbf{\textit{#1}}}}
\newcommand{\AttributeTok}[1]{\textcolor[rgb]{0.77,0.63,0.00}{#1}}
\newcommand{\BaseNTok}[1]{\textcolor[rgb]{0.00,0.00,0.81}{#1}}
\newcommand{\BuiltInTok}[1]{#1}
\newcommand{\CharTok}[1]{\textcolor[rgb]{0.31,0.60,0.02}{#1}}
\newcommand{\CommentTok}[1]{\textcolor[rgb]{0.56,0.35,0.01}{\textit{#1}}}
\newcommand{\CommentVarTok}[1]{\textcolor[rgb]{0.56,0.35,0.01}{\textbf{\textit{#1}}}}
\newcommand{\ConstantTok}[1]{\textcolor[rgb]{0.00,0.00,0.00}{#1}}
\newcommand{\ControlFlowTok}[1]{\textcolor[rgb]{0.13,0.29,0.53}{\textbf{#1}}}
\newcommand{\DataTypeTok}[1]{\textcolor[rgb]{0.13,0.29,0.53}{#1}}
\newcommand{\DecValTok}[1]{\textcolor[rgb]{0.00,0.00,0.81}{#1}}
\newcommand{\DocumentationTok}[1]{\textcolor[rgb]{0.56,0.35,0.01}{\textbf{\textit{#1}}}}
\newcommand{\ErrorTok}[1]{\textcolor[rgb]{0.64,0.00,0.00}{\textbf{#1}}}
\newcommand{\ExtensionTok}[1]{#1}
\newcommand{\FloatTok}[1]{\textcolor[rgb]{0.00,0.00,0.81}{#1}}
\newcommand{\FunctionTok}[1]{\textcolor[rgb]{0.00,0.00,0.00}{#1}}
\newcommand{\ImportTok}[1]{#1}
\newcommand{\InformationTok}[1]{\textcolor[rgb]{0.56,0.35,0.01}{\textbf{\textit{#1}}}}
\newcommand{\KeywordTok}[1]{\textcolor[rgb]{0.13,0.29,0.53}{\textbf{#1}}}
\newcommand{\NormalTok}[1]{#1}
\newcommand{\OperatorTok}[1]{\textcolor[rgb]{0.81,0.36,0.00}{\textbf{#1}}}
\newcommand{\OtherTok}[1]{\textcolor[rgb]{0.56,0.35,0.01}{#1}}
\newcommand{\PreprocessorTok}[1]{\textcolor[rgb]{0.56,0.35,0.01}{\textit{#1}}}
\newcommand{\RegionMarkerTok}[1]{#1}
\newcommand{\SpecialCharTok}[1]{\textcolor[rgb]{0.00,0.00,0.00}{#1}}
\newcommand{\SpecialStringTok}[1]{\textcolor[rgb]{0.31,0.60,0.02}{#1}}
\newcommand{\StringTok}[1]{\textcolor[rgb]{0.31,0.60,0.02}{#1}}
\newcommand{\VariableTok}[1]{\textcolor[rgb]{0.00,0.00,0.00}{#1}}
\newcommand{\VerbatimStringTok}[1]{\textcolor[rgb]{0.31,0.60,0.02}{#1}}
\newcommand{\WarningTok}[1]{\textcolor[rgb]{0.56,0.35,0.01}{\textbf{\textit{#1}}}}
\usepackage{graphicx}
\makeatletter
\def\maxwidth{\ifdim\Gin@nat@width>\linewidth\linewidth\else\Gin@nat@width\fi}
\def\maxheight{\ifdim\Gin@nat@height>\textheight\textheight\else\Gin@nat@height\fi}
\makeatother
% Scale images if necessary, so that they will not overflow the page
% margins by default, and it is still possible to overwrite the defaults
% using explicit options in \includegraphics[width, height, ...]{}
\setkeys{Gin}{width=\maxwidth,height=\maxheight,keepaspectratio}
% Set default figure placement to htbp
\makeatletter
\def\fps@figure{htbp}
\makeatother
\setlength{\emergencystretch}{3em} % prevent overfull lines
\providecommand{\tightlist}{%
  \setlength{\itemsep}{0pt}\setlength{\parskip}{0pt}}
\setcounter{secnumdepth}{-\maxdimen} % remove section numbering
\ifLuaTeX
  \usepackage{selnolig}  % disable illegal ligatures
\fi
\IfFileExists{bookmark.sty}{\usepackage{bookmark}}{\usepackage{hyperref}}
\IfFileExists{xurl.sty}{\usepackage{xurl}}{} % add URL line breaks if available
\urlstyle{same} % disable monospaced font for URLs
\hypersetup{
  pdftitle={Problem Set 2 - Watershed Hidorlogy},
  pdfauthor={Emilio Aguilar},
  hidelinks,
  pdfcreator={LaTeX via pandoc}}

\title{Problem Set 2 - Watershed Hidorlogy}
\author{Emilio Aguilar}
\date{2022-09-18}

\begin{document}
\maketitle

\hypertarget{problem-1}{%
\subsection{Problem 1}\label{problem-1}}

\textbf{1. Clear Lake has surface area of 708,000 m2 (70.8 ha). For a
given month, the lake has an inflow of 1.5 m3/s and an outflow of 1.25
m3/s. A +1.0-m storage change, or increase in lake level was recorded.
If a precipitation gage recorded a total of 24 cm for this month,
determine the evaporation loss (in cm) for the lake. Assume that seepage
(deep drainage) loss is negligible. (10 pts)}

\begin{Shaded}
\begin{Highlighting}[]
\NormalTok{Area\_m2 }\OtherTok{\textless{}{-}} \DecValTok{708000} \CommentTok{\#m2}
\NormalTok{Area\_ha }\OtherTok{\textless{}{-}} \FloatTok{70.8} \CommentTok{\#ha}
\NormalTok{inflow }\OtherTok{\textless{}{-}} \FloatTok{1.5} \CommentTok{\#m3/s}
\NormalTok{outlow }\OtherTok{\textless{}{-}} \FloatTok{1.25} \CommentTok{\#m3/s}
\NormalTok{Storage\_change }\OtherTok{\textless{}{-}} \DecValTok{1} \CommentTok{\#m}
\NormalTok{Precip }\OtherTok{\textless{}{-}} \DecValTok{24} \CommentTok{\#cm}

\CommentTok{\#determine evaporation loss (in cm) for the lake}
\CommentTok{\#Assume that deep drainage loss is negligible }

\CommentTok{\#Area to cm2}
\NormalTok{Area\_cm2 }\OtherTok{\textless{}{-}}\NormalTok{ Area\_m2 }\SpecialCharTok{*} \DecValTok{10000}

\CommentTok{\#Volume of precipitation}
\NormalTok{P\_volume\_cm3months }\OtherTok{\textless{}{-}}\NormalTok{ Area\_m2 }\SpecialCharTok{*} \DecValTok{1000} \SpecialCharTok{*}\NormalTok{ Precip}

\CommentTok{\#Cubic pero second to cubic per month}
\CommentTok{\#Multiply the value of volum/time by 2.628e+12}
\NormalTok{inflow\_cm3month }\OtherTok{\textless{}{-}}\NormalTok{ inflow }\SpecialCharTok{*} \FloatTok{2.628e+12}
\NormalTok{outlow\_cm3month }\OtherTok{\textless{}{-}}\NormalTok{ outlow }\SpecialCharTok{*} \FloatTok{2.628e+12}


\CommentTok{\#Runoff}
\NormalTok{RO\_cm3month }\OtherTok{\textless{}{-}}\NormalTok{ inflow\_cm3month }\SpecialCharTok{{-}}\NormalTok{ outlow\_cm3month}
\end{Highlighting}
\end{Shaded}

Evapotraspiration may be estimated with a water balance approach. if the
change in storage and all the inputs and outputs are known

\[ dV/dt = p(t) + rsi(t) + rgi(t) - rso(t) - rgo(t) - et(t)\] Where, V
is the average volume of water stored, p is precipitation rate, rsi is
average surface water inflow rate, rgi average ground inflow rate rso
average surface water outflow rate, rgo average groundwater outflow rate
and et average transportation. If groundwater flow is negligible we will
have:

\[ dV/dt = p(t) + ΔRO- et(t) \] Solving for et we will have

\[ et(t) = p(t) + RO \] Solving

\begin{Shaded}
\begin{Highlighting}[]
\CommentTok{\#Volume of flow (discharge) }
\NormalTok{dis\_cmmonth }\OtherTok{\textless{}{-}}\NormalTok{ RO\_cm3month}\SpecialCharTok{/}\NormalTok{Area\_cm2}

\NormalTok{et }\OtherTok{\textless{}{-}}\NormalTok{ Precip }\SpecialCharTok{{-}}\NormalTok{ dis\_cmmonth }
\NormalTok{et }
\end{Highlighting}
\end{Shaded}

\begin{verbatim}
## [1] -68.79661
\end{verbatim}

\begin{Shaded}
\begin{Highlighting}[]
\CommentTok{\#Et is 68.7 cm/month}
\end{Highlighting}
\end{Shaded}


\end{document}
